\documentclass[20pt]{article}

\usepackage[utf8]{inputenc}
\usepackage[russian]{babel}
\usepackage{url}
\usepackage[colorlinks,allcolors=blue]{hyperref}
\usepackage[numbers]{natbib}
\usepackage{amsthm}

\newtheorem*{mydef}{Определение}

\title{Дипломная работа}
\author{Козлов Никита}

\begin{document}

{\huge Извлечение данных о состоянии веб-приложения на основе анализа образа памяти веб-сервера}

\newpage

\section*{Аннотация}

Данная работа посвящена исследованию возможности извлечения конфиденциальных данных из слепков памяти веб сервера. Для исследований и экспериментов использовался веб сервер Apache, а в качестве системы управления содержимым сайта использовась система Wordpress. На полученной системе были протестированы
различные методы извлеченя приватных ключей и других конфиденциальных данных.

\newpage

\tableofcontents
\newpage

\section{Введение}

\subsection{Предметная область}

Для работы любого веб приложения необходим веб сервер, который будет обслуживать всех клиентов. В качестве веб сервера обычно вытупает бинарная
программа, так что на данный момент мы можем абстрагироваться от предназначения данной программы и поговорить о том, что такое дампы памяти.

Дамп памяти - это содержимое рабочей памяти процесса. Он может включать в себя значения регистров процессора, содержимое стека и многое другое.
Дамп памяти необходим для отладки программы, так как с помощью него разработчик можем посмотреть состотяние программы на момент ошибки.
С помощью отладчика мы можем посмотреть состояние переменных во всех функциях во всем стеке вызовов.

Во время работы веб сервера ему приходится держать в памяти многие вещи, такие как данные последних запросов клиентов, приватные ключи, логины
и пароли от базы данных, конфигурационные данные и многое другое. Таким образом дамп памяти хранит в себе различную конфиденцеальную информацию.

Дамп памяти может быть получен множеством способов, например если этот самый веб сервер неправильно сконфигурирован и разрешает
чтение файлов из корневой дирректории веб приложения, то атакующий может его скачать, другим спобом получения дампа памяти
является уязвимость Local File Read, которая позволяет читать произвольный файл в системе.

\subsection{Актуальность}
Большое количество неправильно сконфигурированных серверов говорит о том, что существует реальная возможность того, что атакующие
обратят внимание на данную проблему. Также не было найдено утилит, которые позволяют извлекать конфиденциальные данные из дампа памяти веб сервера,
или исследований о том, какую информацию можно найти в слепках памяти.

\section{Цель работы}
Исследовать возможность получения конфиденциальной информации из дампов памяти веб сервера.

\section{Постановка задачи}
На примере веб сервера Apache и сисемы управления содержимым сайта Wordpress исследовать возможность получения конфиденциальной информации
из дампов памяти веб сервера.

\section{Конфиденциальные данные}
Конфиденциальные данные - данные, которые не подлежат разглашению. Такие как cookies, логины, пароли, токены и многое другое.

\subsection{Методы извлечения приватных данных}
Методы извлечения приватных данных

\section{Тестирование}
Тестирование

\subsection{Среда тестирования}
Среда тестирования

\subsection{Результаты тестирования}
Результаты тестирования

\section{Заключение}
Заключение

\bibliography{bibliography}
\bibliographystyle{plainnat}

\end{document}
