\documentclass[20pt]{article}

\usepackage[utf8]{inputenc}
\usepackage[russian]{babel}
\usepackage{url}
\usepackage[colorlinks,allcolors=blue]{hyperref}
\usepackage[numbers]{natbib}
\usepackage{amsthm}

\newtheorem{mydef}{Определение}

\title{Дипломная работа}
\author{Козлов Никита}

\begin{document}

{\huge Извлечение данных о состоянии веб-приложения на основе анализа образа памяти веб-сервера}

\newpage

\section*{Аннотация}

Данная работа посвящена исследованию возможности извлечения конфиденциальных данных из слепков памяти веб сервера. Для исследований и экспериментов использовался веб сервер Apache, а в качестве системы управления содержимым сайта использовась система Wordpress. На полученной системе были протестированы
различные методы извлеченя приватных ключей и других конфиденциальных данных.

\newpage

\tableofcontents
\newpage

\section{Введение}
\begin{mydef}
  Пример определения.
\end{mydef}

\section{Цель работы}
Цель

\section{Постановка задачи}
Постановка задачи

\section{Конфиденциальные данные}
Конфиденциальные данные

\subsection{Методы извлечения приватных данных}
Методы извлечения приватных данных

\section{Тестирование}
Тестирование

\subsection{Среда тестирования}
Среда тестирования

\subsection{Результаты тестирования}
Результаты тестирования

\section{Заключение}
Заключение

\bibliography{bibliography}
\bibliographystyle{plainnat}


\end{document}
